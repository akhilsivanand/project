\chapter{LITERATURE REVIEW}\label{chap2}
\thispagestyle{empty}

Here comes the intro to the literature review done. Tell how the review is classified into sections and subsections, what sources provided the required information about the existing research and the results,.... etc. An important thing to be kept in mind is that preparing the literature review  according to proper subject (topic) classification is a best practice. A reference can be cited in many sub sections or paragraphs where ever it is required to be cited. No citations should be kept without its position in the reference list. This happens when the citations are provided manually by including the appropriate text within the matter. Use the standard practice of keeping a database of references as a \texttt{.bib} file and using the \verb|\cite{key}|, \verb|\citet{key}| or \verb|\citep{key}| code. Those who want to use only \textbf{Microsoft Word}, use either \textbf{EndNote} or \textbf{Mendeley} to keep bibliography database file and to cite the references appropriately. 


\section{Maintanance Management Techniques }
When operating costs of all manufacturing and production plants are considered, maintenance costs cover a major part. It can vary from 15 to 60 percent depending on the type of industry. In food related industries, maintenance costs can be as low as 15 percent, while in a steel or iron plant, maintenance cost can be close to 60 percent. Ineffective maintenance management can resut in lesser quality products and can result in less competitive products, so timely maintenance is an important aspect of any industry.
The main reason for ineffective maintenance is the lack of proper predictive maintenance techniques to predict a need for repair or maintenance pf machinery, equipment and systems.


In general, maintanance management techniques are categorized into 
(a)~Run-to-Failure Management  and(c)~Predicive Maintenance Management

\subsection{Run to failure Management}

The simplest maintenance strategy is to execute run to failure maintenance. Here, machinery are allowed to operate until they break down, at which point maintenance is performed. No maintenance, is performed on the asset until the occurrence of failure event. However, a plan is in place for ahead of the failure, so that the asset can be fixed without causing any production issues.  This strategy is useful for assets that, on breakdown, pose no safety risks and have minimal effect on production.

In run-to-failure approach we, wait for the machine or equipment
failure beforetaking any maintenance action .So, it can be also called a nomaintenance
approach of management. It is actually the most expensive method of
maintenance management as the cost assocated with failure are much higher than maintanance costs. Only very few plants use a run-to-failure management approach.
In almost all instances, plants perform basic preventive tasks such as lubrication,
machine adjustments etc, even in a run-to-failure environment.
However,in the scenario machines and other plant equipment are not
rebuilt, nor are any major repairs are made until the equipment fails. This approach can cause several costs like,
inventory cost, high overtime labor costs, high machine downtime, and low production
availability.
Since no  maintenance is done, a plant that uses
true run-to-failure management must be able to react to all possible failures within the
plant. This reactive method of management forces the maintenance department to
have all extensive spare parts in the inventory. The alternative is to rely on
equipment vendorswho are able to provide immediate delivery of all needed spare parts.
Even if the second option is possible, extra rate of fast delivery 
increases the costs  and downtime required to correct machine failures.
To cope with the impact of  unexpected machine failures, maintenance
crew must also be able to react quickly to all machine failures. 

\subsection{Preventive Maintanance}

Here, maintenance tasks are carried out on the basis of
elapsed time  of operation. Figure illustrates how the statistical
life of a machine varies in time. This curve, called the bathtub curve
indicates that a new machine has a high chance of failure during the first few weeks of operation due to occurance of severa problems associated with setup and installation. After this  period, the probability
of occurance of failure is relatively low for an extended period. After this normal machine
life period, the probability of failure increases sharply with elapsed time. In preventive
maintenance management, machine repairs or rebuilds are scheduled based on the
statistical measures.
The actual implementation of preventive maintenance varies greatly. Sometimes the equipments need only lubrication and minor adjustments.
In preventive maintenance management, It is  assumed that machines will degrade
over time depending on their running conditions.

\begin{figure}
	\centering
	\includegraphics[width=0.7\linewidth]{"picture-files/Bathtub curve"}
	\caption{}
	\label{fig:bathtub-curve}
\end{figure}



\subsection{Predictive Maintananve}

predictive maintenance is monitoring the conditions of machinary, in
an attempt to detect possible problems and to prevent the occurance of a catastrophic failure. The practice of predictive
maintenance involves regular monitoring different conditions of the  machanery like, vibration, input parameters etc, to predict equipment failure so that 
there is maximum interval
between repairs and minimum amount of costs associated with 
machine-train failures.
Predictive maintenance thus,  improves productivity,
product quality, and overall effectiveness of manufacturing and production
plants.
A good predictive maintenance management program uses the most costeffective
tools (e.g., vibration monitoring, thermography, acoustic emission, humidity etc) to obtain the
actual operating condition of critical plant systems and based on this data,
all maintenance activities are scheduled when needed.  predictive maintenance
i greatly reduces the  maintenance costs and failure costs. It also
improves the product quality, productivity, and profitability of manufacturing and
production plants.
Predictive maintenance can be called as a condition-driven preventive maintenance program. Most
mechanical problems can be minimized if they are detected and repaired at the right time.That is , If
the problem is detectedat the right time, major repairs can usually be prevented.

Predictive maintenance using vibration measurement depends on two basic criteras: (1)  common failure modes have distinct  vibration frequency components which stay away from the custer of data points,
that can be isolated and identified, and (2) the amplitude of each vibration
component will remain constant unless the operating conditions of the machinetrain, like operating speed and load
changes.



\section{Mechanical Condition Monitoring Techniques}

The different techniques used for assessing the condition of machinery are given below

\subsection{Vibration Analysis}

This approach is very popular and well accepted in plants to detect faults at an early stage. As a general rule, machine does not break down or fail without some form of warning, which is indicated by an increasing vibration level. The frequency range typically from approximately 1 Hz to near 20 kHz.This technique is used mainly to track mass unbalance, bent shaft,  Misalignment and preloads, Crack, Fluid induced instability, Mechanical looseness and Bearing assembly looseness.The main advantage associated with vibration analysis is that, it is capable to diagnose an extensive range of faults or failure in rotating machinery and is very effective at detecting resonance. The main limitation associated wit this techinique is that, it is unable to monitor low speed machinery.

\subsection{Ultrasonic inspection }

In this approach, recorded high frequency emissions are electronically translated down and analyzed for enhanced diagnostics. Since ultrasound wave lengths are magnitudes smaller, this approach is much more conducive to locating and isolating the source of problems in loud plant environments. Related frequency ranges from 20 kHz to approximately 80 kHz.
This technique is used mainly to track leak detection, bearing condition, lack of lubrication, over lubrication, ionization, cavitations, fault analysis of compressor's valve and Testing for arcing and corona in electrical apparatus etc. 
The main advantage associated with ultrasonic measurement is that it tends to be highly localized, capable for operating in loud and noisy environments, capable to be used in slow-speed machines and has a lower cost in contrast with vibration analysis. 
Its main disadvantage is limited diagnostic ability in comparison with vibration analysis, more reliable result is achievable when it is used as a complementary approach in an integrated package of conditioning monitoring techniques.

\subsection{Thermography }

Thermography is the process of using a thermal imager to detect emitted heat of objects. This technology allows operators to validate normal operations and, locate thermal anomalies which indicate possible faults.
It is used mainly to track any fault that lead to temperature increase in components including: Friction unbalance, Shaft bent or bow, Misalignment and preloads, Bearing assembly looseness, Unsuitable lubrication and Electricals faults. 
It is a Simple, quick and efficient screening tool since it uses non-contact remote sensing. The main disadvantage assoiated with thermography is late warning of impending failure in comparison to vibration based methods. Likewise as extensive working experience about the faulty equipment and sufficient heat transfer knowledge is required to utilise this method for predictive maintanance.

\subsection{Acoustic emission}

Acoustic emissions are the sound waves or stress waves generated when a piece of material undergoes stress due to external forces. These waves can be measured to detect where the stress has caused wear or degradation, including crushing, cracking or any kind of impacts.It is usually used
 to track friction and wear, Leakage, Lack of lubrication, bearing assembly looseness, cracking, spalling and cyclic fatigue.
It has an earlier detection rate in comparison to vibration analysis and has no spectral overlap with mechanical vibration, Likewise it is not affected by the mechanical noise from adjacent machinery or structural resonances and Only one AE sensor is sufficient. 
The only problem associated with this techique is that th signal attenuation may affect the results and can be difficult to process, interpret and classify the intelligent information from the acquired AE data.



\section{Data Acquisition System}\label{DAQ}
Data Acquisition is the process of measuring and analysing various electrical and physical entities like voltage, vibration, temperature, pressure etc. A DAQ system consists of sensors, signal conditioning circuitry, analog to digital converter, and application software. DAQ systems has several applications which include Research and Analysis, Control and automation, Design validation and Verification etc. DAQ's applications are not only limited to medical instruments, industrial equipment and other home appliances but are used for a variety of products.
A simple  data acquisition system usually consists of an arduino board connected with required sensors depending on the type of input required.

\subsection{Ardunio UNOR3}

\begin{figure}
	\centering
	\includegraphics[width=0.7\linewidth]{picture-files/Arduino}
	\caption{Ardunio UNOR3}
	\label{fig:arduino}
\end{figure}

The Ardunio UNO is a microcontroller board
based on ATmega328. It has 14 digital
input/output pins(of which 6 can be used as
PWM outputs),6 analog inputs.\citep{Arduino}. It also has a 16MHz ceramic resonator,a USb connection
a power jack, an ICSP header and a reset
button.

\subsection{Wifi Module}

The WiFi module used in our system will help
us to operate the web page for a customer.We can set a particular threshold
vaue to limit the meter reading through these
which will be interfaced with the help of
MAX232 to ardunio UNO board.



\begin{figure}
	\centering
	\includegraphics[width=0.7\linewidth]{"picture-files/wifi module"}
	\caption[Wifi Module]{Wifi Module}
	\label{fig:wifi-module}
\end{figure}



\section{Indirect purchases}
According to Neef (2001) procurement materials can be divided into two separate categories: direct and indirect. Direct materials are those involved in the manufacturing process and related to the production of finished goods, whereas indirect materials relate to the materials that do not result directly in finished goods.
 

Telgen and de Boer (1995) identified the typical characteristics related to indirect purchases: \\
(1) They consist of a wide range of goods and services, which are often purchased from an even larger number of suppliers.\\
(2) They are often time consuming as they consist of non-standardized items which are usually purchased in small orders.\\
(3) They show high end user involvement in the tactical purchasing phases which implies that indirect purchasing takes place virtually all over the firm.\\
(4) In total a lot of money is involved in indirect purchases, and \\
(5) they attract low attention from managers. 

Due to the varying characteristics of purchasing indirect materials, buyers often have to spend a lot of time dealing with individual transactions. This means negotiating with suppliers, converting purchase requests to purchase orders, handling queries and ensuring the correct allocation of invoices received. This huge operational workload is time consuming and derives buyers to neglect more strategic tasks \citep{pusch2005}. % (Puschmann \& Alt 2005) 

\section{Procurement Process }
The procurement process is one of the most important processes of a company. The procurement process usually varies between companies due to activity times and relations with suppliers \citep{trk2010}.
 A basic procurement process starts with the specification of need and ends with settlement and payment. Presutti (2002) states that e-procurement systems have the power to transform the purchasing process because it has an effect on all of the steps identified. 

E-procurement brings about important simplifications of the operational workload for buyers by decentralizing the operational procurement process, therefore improving the effectiveness and efficiency of the purchasing process and enabling buyers to focus on more strategic tasks (Presutti 2002; Puschmann \& Alt 2005). When companies are adopting e-procurement solutions one has to remember that organizational changes (and / or process improvements) can often bring even greater savings than implementation of a simple technology (Trkman and McCormack 2010). 

Kalakota and Robinson (2001, s.308) have listed the five key challenges Procurement managers are facing in the increasingly competitive business world: \\
- Reducing order processing cost and cycle times 
- Providing enterprise-wide access to corporate procurement capabilities \\
- Empowering desktop requisitioning through employee self-service \\
- Achieving procurement software integration with company's back office systems \\
- elevating the procurement function to a position of strategic importance within the organization \\
E-procurement can help companies to achieve the targets listed above. It can have an impact on the whole procurement function and its processes, as well as other corporate business functions for example accounting. Next, the benefits and challenges of implementing an e-procurement solution are examined.

\section{Implementation of E-procurement}
Implementing an E-procurement solution is not as simple as many businesses think \citet{croom2005,angels2007}.  According to Yu, Yu, Itoga and Lin (2008) companies implementing e-procurement need to clearly understand the purpose of launching such a system. It involves careful analysis about how e-procurement will affect a company and its strategy and in which area it will obtain financial and non-financial benefits. The drivers and problem factors behind adopting E-procurement technologies vary between companies, when businesses are adopting e-procurement solutions there are several factors to consider on many levels of the organization. 

To succeed in e-procurement implementation Kalakota and Robinson (2001, s.337-347) have proposed a seven step roadmap for business managers. The roadmap starts with clarification of goals and ends to the education of solutions end-users. According to the authors all of the steps need to be covered thoroughly in order to fully succeed in e-procurement implementation. Clarify your goals: Businesses should make sure that the business problem or goal is well defined and understood. Procurement managers need to ask themselves what are the functions you are trying to improve and are the goals clearly defined and reachable Construct a process audit: After setting the goal businesses should analyse their current procurement process. It is important to understand where you are now, in order to reach the tomorrow. Businesses should first determine what kind of purchasing is the solution targeted to support: direct or indirect (Kalakota \& Robinson 2001). %\citet{bijulal2015}

As Presutti (2002) states, for a business to realize maximum value from an E-procurement initiative, the whole purchasing process must be evaluated to determine if it needs to be re-engineered. Create a business case for e-procurement: Setting up a business case for E-procurement implementation can be useful, as it forces the company to systematically analyse the business (Kalakota \& Robinson 2001).

 Smart (2010) recognizes that there has been a problem in measuring the value of IT investments and in building a business case for such investments. This derives from the fact that, in many cases the benefits from implementing an E-procurement solution are intangible and non-financial therefore some traditional accounting based-methods such as ROI are not able to capture them (Piotrowicz \& Irani, 2010). 
Develop a supplier integration matrix: Without supplier commitment and involvement, the e-procurement project is useless. Companies should develop a supplier integration matrix. The matrix helps determine what kind of relationship is best for individual vendors \citep{kala2001}.  (Kalakota \& Robinson 2001). 
Involving suppliers in organizations e-procurement deployment is important, since it also has a significant impact on suppliers IT-infrastructure and strategy \citep{croom2005}. %(Croom and Brandon-Jones, 2005). 
As Smart (2010) identified, neglecting the impact of suppliers in company's e-procurement deployment may lead to the failure of the whole project. Select an e-procurement application: There is a variety of different e-procurement applications for companies to choose from (de. Boer et al., 2002).  By categorizing the products and services purchased, companies can more easily decide on the required procurement strategies and e-procurement applications (Smeltzer, 2001).  

%Kalakota and Robinson (2001) 
\citet{kala2001} suggest four questions that managers should think about, in order to define the right application for their company: Will it support my procurement process; does it leverage my other application investments; will it work seamlessly with other applications and; is it extendible? Remember: integration is everything: Integrating the e-procurement solution with suppliers and company's existing back-office systems is the most important thing in e-procurement implementation \citep{kala2001}.  %(Kalakota \& Robinson 2001).
 According to \citet{croom2005} % Croom and Brandon-Jones (2005) 
 Integration with company's finance system had a direct impact on the level of process savings and was also an important determinant in selecting the application. Educate, educate, educate: Redesigning the procurement process and influencing end-user behavior towards the new procedures and business rules is one of the 18 most critical factors in a successful e-procurement implementation \citep{angels2007}. %(Angeles and Nath, 2007). 
Change tends to generate resistance and managers should deal with it by communicating and encouraging employees to comply with the new guidelines \citep{kala2001}. %(Kalakota \& Robinson 2001).
\citet{angels2007} % Angles and Nath (2007) 
propose that providing information about their spend to employees encourages them to take ownership of savings targets with the use of re-engineered procurement processes. 

\section{Benefits of E-procurement }
The benefits of adopting e-procurement technologies have been widely researched in the literature (Kalakota and Robinson 2001; Attaran \& Attaran 2002; de Boer et al. 2002; Davila et al. 2003; Croom and Brandon-Jones 2005). 
The primary motivation for companies adopting e-procurement solutions has been cost reductions and process efficiencies. 
Croom and Brandon-Jones (2005) found that cost reductions in goods purchased comprise from three key issues: consolidation of purchase specifications; reducing the number of suppliers and; through improved compliance with existing contracts. 
A research by Quesada et al. (2010)proposes that E-procurement technologies affect positively to company's procurement practices and procurement performance. Positive impact on procurement practices facilitates the development of operational tasks in the procurement function, which leads to continuous improving. As the operational tasks are performed more effectively the procurement performance is enhanced. 

According to Davila et al. companies using e-procurement solutions report savings of 42 percent in purchasing transactions costs. Another research by Croom and Johnston (2003) found that E-procurement implementation can have up to 75\% cost reduction in procurement process costs and 16 - 18 \% reduction in purchasing price for indirect purchases.

 According to Croom and Brandon-Jones (2005) complying with existing contracts is an important mechanism for realizing lower prices and discounts. The savings that come out from automating the process derive from eliminating paperwork and human intervention, reducing transaction costs and cycle time and also from streamlining and automating the audit trail and approval process (Neef, 2001 s.48).
 
 While the cost savings can be significant, de Boer et al. (2002) argue that the total volume of purchases needs to be high, as well as the amount of internal customers, in order to reach savings as high as mentioned above. 
The research by Davila et al. (2003) also identifies that companies using e-procurement gain additional control over maverick spending and can reduce the headcount supporting purchasing transactions. 

To support this Croom and Johnston (2003) found that e-procurement can have a major impact on compliance on many different levels of the procurement  process: it supports managerial budgetary control; reduces data entering failures; offers greater transparency and accessibility to corporate wide spending; improves system reliability; and improves the access to managerial information.



\subsection{E-procurement process risks}

This risk relates to the security and control of the E-procurement process itself. Such issues can be related to, for example data security and fraud prevention e.g. fake suppliers, fake bids etc.  As identified in the examination of earlier e-procurement literature, adopting E-procurement solutions can provide substantial cost savings and other benefits, but there are also challenges and risks companies need to take into account when considering e-procurement adoption. Making the procurement process more efficient and faster can be achieved with the use of e-procurement solutions. Nonetheless, this requires that the implementation process must be planned and executed thoroughly in order to minimize the challenges and risks companies might face. While indirect purchases can sometimes account for a big part of company's overall spending it is important that also these purchases are conducted following company policies and instructions. Using \index{e-procurement}e-procurement only for indirect purchases in the beginning can act as stepping stone for companies before moving into comprehensive e-procurement which also involves direct purchases.

\section{Relevance of Literature Reviewed}

Use this section to write the relation between the problem being analysed and the relevance of the reviewed literature. Also, relate the concepts, tools and theory understood from the literature review to the research problem being discussed in this report.

\section{Research Methodology}\label{sec-rm}

This section introduces the techniques and tools used for project work, especially, methodology and sample selection, research design, period of the study, sources of data, tools of data collection, tools for data analysis, statistical analysis, broad hypotheses put for testing, limitations, etc. This title resembles the section in Chapter 1, Section~\ref{sec-rm-intro}. However, here the difference is that, the methodology has to be correlated with the literature review done in this chapter. In the previous chapter you will provide only an overview of the methods adopted in the project work.

%\subsection{Research Design}

Here you have to provide the design of the project work, setting hypothesis and hypothesis testing tools used, etc., data collection and collection methods, etc., under different sub-sections.

\subsection{Data Collection}

What are the sources of data for the work, how it was collected, type of source, etc. have to be discussed here. 

\subsection{Hypotheses}
What hypothesis are to be set to achieve the objectives have to be discussed here. If there are multiple hypotheses related to different aspects, provide each of them with appropriate assumptions to be used. 

\subsection{Tools for Data Analysis}

All the tools for the data analysis have to be provided in this section.

\subsection{Tools for Hypothesis Testing}

\section{Summary}

This is a must especially in this chapter, which will tell the reader what are the points you accepted for the analysis and which forms the basis for the study.