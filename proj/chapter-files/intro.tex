\chapter{INTRODUCTION}\label{chap1}
\thispagestyle{empty}

In every chapter we usually provide an introduction to the chapter in this space. Since it is the chapter named `Introduction', a brief introduction about the project work is to be provided here in this chapter. It can extend to two or three pages, if required. This unnamed section can even hold citations to references also.

After reading this section of the report the reader will get an idea about the problem being discussed in this report and the tools used to solve/analyse the problem and address the research question. 

\section{Problem Definition}

Provide a brief description of the problem and its domain, the practical importances of the problem, etc.


\section{Objectives of the Project Work}

The objectives of this project work are:
\begin{enumerate}
\item  To measure the cost effectiveness of E-Procurement of indirect materials in BPCL-KR.
\item  To compare the lead time of traditional procurement and E- procurement of indirect materials.
\item  To compare the cost pattern of traditional procurement and E-procurement.
\item To assess the role of E- procurement in market penetration.
\item To measure the role of E-procurement in inventory management.
\end{enumerate}



\section{Scope of the Project Work}

The scope of this project work to be included in this section. By the term `scope' we usually intend to provide the boundary of the problem and the validity and applicability  of: (a)~the results based on the volume and source of data, (b)~the methodology used to solve the problem, (c)~the sensityvity of the results based on the parameter settings, etc.


\section{Research Methodology}\label{sec-rm-intro}

This section is to provide an overview of the selection of research methods used in this project work, tools, variables, hypothesis,  especially with the data collection, data analysis, making inferences, testing of the hypotheses, comparing the results, etc. This section  should provide only an introductory description. Description of each has to be provided in detail after the literature review in the next Chapter, Sec~\ref{sec-rm}.



\section{Limitations of the Project Work}
As the title says, this section is dedicated to explain what limitations exist for the project work in terms of the validity of the results because of the method used, data source, data collection method, difficulties faced in different stages of the project, etc. It can go up to two paragraphs.

Now here you can provide a last paragraph for chapterisation. A sample will look like the following.

Refer the chapters by their labels as follows. Chapter~\ref{chap2} (page~\pageref{chap2}) discusses the  ........ The data collection method and the details of data sources are provided  Chapter~\ref{chap3} (page~\pageref{chap3}),    etc... The conclusions of the findings are provided in Chapter~\ref{conc} (page~\pageref{conc}). It should be sequentially and logically framed such that the reader can decide which chapter will be of importance to him and can turn into that chapter.
