\chapter{DATA ANALYSIS}\label{chap4}
\thispagestyle{empty}

In Chapter~\ref{chap3}, a figure was inserted to show the capabilities of \LaTeX. Now we can refer back to that figure like: Fig.~\ref{figureone}, Page no.~\pageref{figureone}.


Like a figure, we can insert a table and refer it anywhere in the document. We had a table in page no.~\pageref{tab-exp}, numbered as Table~\ref{tab-exp}. The following is another sample table. Provide captions for every table to enable readability. Usually the table captions are provided above the table, as in Table~\ref{tableone}

\begin{table}[h]\centering \caption[First table]{First sample table with the table caption above the table}\label{tableone}
\begin{tabular}{|l|c|r|}
\hline Left align & Center & Right align\\
\hline one  & two &  three\\\hline 
four & five  & six \\
\hline 
\end{tabular}
\end{table}

\section{Equations}

We can have many types of equations. They are single equation, equation array, and aligned equations. The first one below is a single equation.
\begin{equation}
p(x\leq n)=\sum_{i=0}^{n} \frac{e^{-\lambda x} {(\lambda x)}^i}{i !}\label{poiss}
\end{equation}

Now we can see an equation array.

\begin{eqnarray}
f(x)=\lambda e^{-(\lambda x)} \quad \text{pdf of exponenial}\label{exp}\\
S=ut+\frac{1}{2}at^2\label{dist}
\end{eqnarray}

The above equations are aligned to the right. We can make them aligned at any character. If we select the equal sign as the alignment position, we have to use align environment like this.

\begin{align}
f{(x)}&=\lambda e^{-(\lambda x)} \quad \text{pdf of exponenial}\label{exp1}\\
S&=ut+\frac{1}{2}at^2\label{dist1}
\end{align}

\section{Using the equation, table and figure  numbers globally}

The above equations can be referred to at any position in the document. It is by its identifiers. Eqn.~\eqref{dist1} measures the distance an object travels in time $ t $, starting with an initial velocity $ u $ and an acceleration $ a $. Eqn.~\eqref{poiss} gives the cumulative probability of a Poison process that there will be $ n $ or less events in a given period of $ x $ units of time when the process has an average rate of $ \lambda $ per time.

In the same way we can refer any table or figure in the document at any place. Example, Table~\ref{tab:side} is a sideways table, placed alone in a page. 


\section{Summary}